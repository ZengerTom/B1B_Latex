\chapter{Schrift}
\LaTeX{} unterstützt die Verwendung von unterschiedlicher Schriftarten. Die Voreinstellungen sind 10 Punkte Schriftgröße (1 Punkt entspricht ca. 0,35mm) und Roman Schrift (mit Serifen).
\section{Schriftfamilie}
In \LaTeX{} stehen drei verschiedene Schriftfamilien zur Verfügung.
\begin{table}[H]
\centering
\begin{tabular}{| m{5cm} | m{5cm} |}
\hline
\multicolumn{1}{|c|}{Proportionalschrift} 	&\multicolumn{1}{|c|}{Nichtproportionalschrift}\\
\hline
\begin{itemize}
\item Serif
\item Sans Serif
\end{itemize}
& \begin{itemize}
\item Type Writer\end{itemize}\\
\hline
\end{tabular}
\end{table}
Diese können mit folgenden Befehlen deklariert werden. Welche Schrift dabei gewählt wird hängt von den Standardeinstellungen (Installation, Dokumentenklasse, Präambel) ab.
\begin{center}
\begin{tabular}{lll}
\verb=\textrm{text}=&\textrm{Schrift mit Serifen}&\verb=\rmfamily=\\
\verb=\textsf{text}=&\textsf{Schrift ohne Serifen}&\verb=\sffamily=\\
\verb=\texttt{text}=&\texttt{Schreibmaschinenschrift}&\verb=\ttfamily=\\
\end{tabular}
\end{center}
\section{Schriftgröße}
In \LaTeX spezifizieren folgende Befehle die Schriftgröße. Dies erfolgt relativ zur Grundschriftgröße. Eine Änderung der Schriftgröße kann auch mit Hilfe einer Umgebung erfolgen.
\begin{tabbing}
xxxxxxxxxxxxxxxxxxxxxxxxxx\=\kill
\verb=\tiny=			\>\tiny{Schriftgröße}\\
\verb=\scriptsize=		\>\scriptsize{Schriftgröße}\\
\verb=\footnotesize=		\>\footnotesize{Schriftgröße}\\
\verb=\small=			\>\small{Schriftgröße}\\
\verb=\normalsize=		\>\normalsize{Schriftgröße}\\
\verb=\large=			\>\large{Schriftgröße}\\[3pt]
\verb=\Large=			\>\Large{Schriftgröße}\\[3pt]
\verb=\LARGE=			\>\LARGE{Schriftgröße}\\[3pt]
\verb=\huge=			\>\huge{Schriftgröße}\\[3pt]
\verb=\Huge=			\>\Huge{Schriftgröße}\\[3pt]
\end{tabbing}
\section{Schriftstil \& Hervorhebung}
Außer der Schriftfamilie hat ein Font die Attribute Schriftstärke (\textit{series}) und Schriftform (\textit{shapes}). Die Befehle zur Änderung der Schriftstärke beeinflussen die Stärke der Linien und die Laufweite (\textit{weight \& width}). Bei Änderung der Schriftform wird die Ausrichtung der Buchstaben geändert.\\
\\
\begin{tabular}{ll}
\multicolumn{2}{l}{Series}\\
\verb=\textbf{text}=&\textbf{Fette Schrift}\\
\verb=\textmd{text}=&\textmd{Normale Schrift}\\
\\
\multicolumn{2}{l}{Shapes}\\
\verb=\textup{text}=&\textup{Aufrechte Schrift (upright)}\\
\verb=\textsl{text}=&\textsl{Schräge Schrift (slanted}\\
\verb=\textit{text}=&\textit{kursive Schrift (italic)}\\
\verb=\textsc{text}=&\textsc{Schrift in Kapitälchen (smallcaps)}\\
\verb=\textnormal{text}=&\textnormal{Grundschrift des Dokuments}\\
\\
\end{tabular}
\\
Man kann \LaTeX{} auch die Standardhervorhebung überlassen. \LaTeX{} ermittelt dann selbstständig welche Hervorhebungsform verwendet wird. Hierzu wird der \verb=\emph=-Befehl verwendet.\\
\begin{lstlisting}[caption={Emph-Hervorhebung}, escapechar=@, breaklines=true]
\textit{Lorem ipsum dolor sit amet, consetetur sadipscing elitr, sed
diam \emph{nonumy} eirmod invidunt ut labore et dolore magna aliquyam.}

@\colorbox{lstback}{\textit{Lorem ipsum dolor sit amet, consetetur sadipscing elitr, sed diam \emph{nonumy} eirmod}}@
@\colorbox{lstback}{\textit{invidunt ut labore et dolore magna aliquyam.}}@
\end{lstlisting}
