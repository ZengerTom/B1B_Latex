\chapter{Titelseite}
\LaTeX{} stellt Methoden zur Verfügung eine Titelseite für das Dokument zu erstellen. Im Top Matter (direkt nach \verb=\begin{document}=) können Informationen hierzu bereitgestellt werden.
\begin{table}[H]
\begin{tabbing}
xxxxxxxxxxxxxxxxxxxxxxxxxxxxxx\=\kill
\verb=\title{titel}=					\>Setzt Titel\\
\\
\verb=\date{datum}=					\>Setzt Datum\\
								\>Leeres Argument, falls kein Datum gewünscht\\
\\
\verb=\today=						\>Aktuelles Datum\\
\\
\verb=\author{}=						\>Setzt Autor\\
\\
\verb=\and=						\>mehrere Autoren\\
\\
\verb=\thanks{}=						\>Danksagung, Fußnote\\
\\
\verb=\maketitle=					\>Erstellt Titelseite\\
\end{tabbing}
\caption{Titelseite (Befehle)}
\end{table}
Mit Hilfe der Umgebung \texttt{titlepage} kann man eigene Titelseiten erstellen. Dabei muss allerdings das komplette Layout selbst angepasst werden. Das Paket \textsl{titling} ist zum Anpassen des \texttt{maketitle}-Befehls geeignet.