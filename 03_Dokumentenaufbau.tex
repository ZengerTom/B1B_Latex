\chapter{Dokumentenaufbau}
\section{minimaler Dokumentenaufbau}
\paragraph{\textcolor{blue}{\textbackslash documentclass}}\leavevmode \\
Legt fest in welcher Art und Weise ein Dokument erzeugt wird. Hier festgelegte Eigenschaften gelten für das gesamte Dokument.
\paragraph{\textcolor{blue}{\textbackslash begin}\{document\}}\leavevmode \\
Mit diesem Befehl wird die Dokumentenumgebung eingeleitet. Es folgen der eigentliche Text des Dokumentes und die Formatierungsbefehle.
\paragraph{\textcolor{blue}{\textbackslash end}\{document\}}\leavevmode \\
Beendet die Dokumentenumgebung. \LaTeX wird alle nachfolgenden Eingaben ignorieren.
\begin{lstlisting}[caption={Minimaler Dokumentenaufbau}]
\documentclass{article}
\begin{document}
	Hallo minimaler Aufbau!
\end{document}
\end{lstlisting}
\section{Deklaration Dokumentenklasse}
\begin{verbatim}
\documentclass[<optionen>][<klasse>]
\end{verbatim}
Es muss genau eine Dokumentenklasse angegeben werden. Optionen dürfen mehrere angegeben werden. Diese sind durch Kommata zu trennen.
\section{Standard-Dokumentenklassen}
In der Standardinstallation sind bereits einige Grunddokumententypen enthalten. Dokumententypen des Koma-Script-Pakets passen die Dokumentenklassen an die Bedürfnisse der deutschen/europäischen Nutzern an.
\paragraph{Klassen}
\begin{tabbing}
xxxxxxxxxxxxxxxxx\=\kill
\texttt{article}		\>Dokumentklasse für kurze, technische Artikel\\
\texttt{book}		\>Dokumentklasse für das Schreiben von Büchern\\
\texttt{report}		\>Dokumentklasse für längere technische Artikel\\
				\>Zwischenstufe zwischen \textsl{article} und \textsl{book}\\
\texttt{letter}		\>Dokumentklasse zum Schreiben von Briefen\\
\texttt{scrartcl}		\>KOMA-Script Äquivalent zu \textsl{article}\\
\texttt{scrbook}		\>KOMA-Script Äquivalent zu \textsl{book}\\
\texttt{srcreprt}		\>KOMA-Script Äquivalent zu \textsl{report}\\
\texttt{scrlttr2}		\>KOMA-Script Äquivalent zu \textsl{letter}\\
\texttt{beamer}		\>Dokumentklasse für Präsentationen\\
\texttt{moderncv}	\>Lebensläufe\\
\end{tabbing}
\section{Gliederungsebenen}
\setlength{\tabcolsep}{4mm}
\renewcommand{\arraystretch}{2}
\begin{table}[H]
\centering
\begin{tabular}{|c|c|c|c|c|}
\hline
\textbf{article}				&\textbf{book}				&\textbf{report}				&\textbf{letter}				&\textbf{Level}\\
\hline
&\textbackslash chapter		&\textbackslash chapter		&&0\\
\textbackslash section			&\textbackslash section		&\textbackslash section		&\textbackslash signature	&1	\\
\textbackslash subsection		&\textbackslash subsection		&\textbackslash subsection		&\textbackslash address	&2	\\
\textbackslash subsubsection	&\textbackslash subsubsection	&\textbackslash subsubsection	&\textbackslash opening	&3	\\
\textbackslash paragraph		&\textbackslash paragraph		&\textbackslash paragraph		&\textbackslash closing	&4	\\
\textbackslash subparagraph		&\textbackslash subparagraph	&\textbackslash subparagraph	&					&5\\				
\hline
\end{tabular}
\caption{Gliederungsebenen der Klassen}
\label{Tab.1}
\end{table}
\section{Optionen der Standardklassen}
Einige Beispiele für Optionen. Genaueres ist in den Klassendokumentationen nachzulesen.
\begin{tabbing}
xxxxxxxxxxxxxxxxxxxxxxxxxx\=\kill
\textbf{Papiergröße}\> letterpaper, a4paper\\
\textbf{Titelblatt}\>titlepage, notitlepage\\
\textbf{Schriftgröße}\>10pt, 11pt, 12pt\\
\textbf{Papierausrichtung}\>portrait, landscape\\
\textbf{Drucklayout}\>oneside, twoside\\
\textbf{Spaltenlayout}\>onecolumn, twocolumn\\
\textbf{Math. Gleichungen}\>fleqn, leqno
\end{tabbing}
