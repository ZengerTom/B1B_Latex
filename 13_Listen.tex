\chapter{Listen}
In \LaTeX{} gibt es mehrere Möglichkeiten unterschiedliche Listen zu realisieren. Hierzu gibt es verschiedene Umgebungen.

\section{einfache Listen}
Mit der Umgebung \verb=itemize= können einfache Listen erzeugt werden. Eine Verschachtelung ist möglich.

\begin{lstlisting}[caption={\texttt{itemize}-Liste}]
\begin{itemize}
	\item
	\item
\end{itemize}
\end{lstlisting}

\section{Numerierte Listen}
Mit der Umgebung \verb=enumerate= können nummerierte Aufzählungen erzeugt werden. Eine Verschachtelung ist möglich.

\begin{lstlisting}[caption={\texttt{enumerate}-Liste}]
\begin{enumerate}
	\item
	\item
\end{enumerate}
\end{lstlisting}
\newpage
\section{Beschreibung}
Die Umgebung \verb=description= eignet sich für Beschreibungen.

\begin{lstlisting}[caption={\texttt{itemize}-Liste}]
\begin{description}
	\item [text:] beschreibung
	\item [text:] beschreibung
\end{description}
\end{lstlisting}

\begin{figure}[H]
\begin{multicols}{3}
\textbf{einfache Liste}
\begin{itemize}
	\item Punkt1
	\item Punkt2
	\item  Punkt3
		\begin{itemize}
		\item Punkt3.1
	  	\item Punkt3.2
		\end{itemize}
	\item Punkt4
\end{itemize}
\columnbreak
\textbf{nummerierte Liste}
\begin{enumerate}
	\item Punkt1
	\item Punkt2
	\item  Punkt3
		\begin{enumerate}
		\item Punkt3.1
	  	\item Punkt3.2
		\end{enumerate}
	\item Punkt4
\end{enumerate}
\columnbreak
\textbf{Beschreibung}
\begin{description}
	\item [C:]		Beschreibung
	\item [JAVA:]	Beschreibung
	\item [PHP:]	Beschreibung
	\item [HTML:]	Beschreibung
\end{description}
\end{multicols}
\caption{Listen Beispiele}
\end{figure}