\chapter{Umbruch}


\section{Blocksatz}
Der Standard-Satz in \LaTeX{} ist der Blocksatz. Hierbei wird der Rand ausgeglichen, Umbrüche werden gesetzt und eine automatische Aufteilung der Wörter wird vorgenommen. Falls nötig erfolgt auch die Trennung eines Wortes.
\section{Trennungshilfen}
\LaTeX{} übernimmt normalerweise automatisch die Trennung von Wörtern. Bei unbekannten Wörtern oder Fremdwörtern kann das allerdings zu fehlerhafter Trennung führen. Durch das Setzen von Trennhilfen kann man \LaTeX{} mitteilen, an welcher Stelle ein Wort getrennt werden soll oder eine Trennung explizit untersagen.
\begin{table}[H]
\begin{tabbing}
xxxxxxxxxxxxxxxxxxxxxxxxxxxxxx\=\kill
\verb=\hyphenation{trenn-ung}=			\>Wort wird an angegebener Stelle getrennt\\
								\>Deklaration in der Präambel\\
\\
\verb=\-=							\>Trennung erfolgt nur an dieser Stelle (einmalig)\\
\\
\verb=''-=							\>Trennung darf auch an anderen Stellen erfolgen\\
								\>(Paket \textsl{babel}\\
\\
\verb=\mbox{wort}=					\>Wort wird nicht getrennt\\
								\>Es erfolgt keine Anpassung der Wortabstände\\
\\
\verb=~=							\>Es erfolgt keine Trennung (hartes Blank)\\
\end{tabbing}
\caption{Trennungsbefehle}
\end{table}


\section{Zeilenumbruch \& Absatz}
\LaTeX{} organisiert den Text in Absätze welche vertikal auf den Seiten platziert werden. Folgende Befehle dienen nicht dazu das Design zu verbessern. Hierzu gibt es Möglichkeiten die Abstände anzupassen.
\begin{table}[H]
\begin{tabbing}
xxxxxxxxxxxxxxxxxxxxxxxxxxxxxx\=\kill
\verb=\\[vAbstand]=					\>Zeilenumbruch, zwangsweise\\
								\>Kein Absatzende)\\
\\
\verb=\\*[vAbstand]=					\>Verhindert einen Zeilenumbruch\\

\\
\verb=\newline=						\>Erzeugt einen Zeilenumbruch\\
								\>Kein Absatzende\\
\\
\verb=\linebreak[Priorität04]=			\>möglicher Zeilenumbruch\\
								\>Zeile vor Umbruch bis zum Rand gestreckt.\\
\\
\verb=\nolinebreak[Priorität04]=			\>möglichst keinen Zeilenumbruch\\
\\
\verb=Leerzeile=						\>Neuer Absatz\\
\\
\verb=\par=						\>Neuer Absatz\\
\end{tabbing}
\caption{Umbruch (Befehle)}
\end{table}


\section{Seitenumbruch}
Wie schon erwähnt kümmert sich \LaTeX{} selbstständig um etwaige Umbrüche von Zeilen. Selbiges trifft ebenfalls auf Seitenumbrüche zu. Auch hier kann man dies mit den entsprechenden Befehlen manipulieren.

\paragraph{Automatische Steuerung}\leavevmode
\begin{table}[H]
\begin{tabbing}
xxxxxxxxxxxxxxxxxxxxxxxxxxxxxx\=\kill
\verb=\flushbottom=					\>Vertikaler Ausgleich an, alle Seiten gleich lang\\
\\
\verb=\raggedbottom=				\>Vertikaler Ausgleich aus\\
\end{tabbing}
\caption{Automatischer Seitenumbruch}
\end{table}
\newpage
\paragraph{Manuelle Steuerung}\leavevmode
\begin{table}[H]
\begin{tabbing}
xxxxxxxxxxxxxxxxxxxxxxxxxxxxxx\=\kill
\verb=\newpage=					\>Seitenumbruch, zwangsweise\\
								\>Spaltenumbruch, twocolumn-Dokumenten\\
\\
\verb=\clearpage=					\>Seitenumbruch, zwangsweise\\
								\>Floating Umgebungen werden eingefügt\\
\verb=\pagebreak[Priorität04]=			\>möglicher Seitenumbruch\\
								\>Seite vor Umbruch bis zum Rand gestreckt\\
\\
\verb=\nopagebreak[Priorität04]=			\>möglichst keinen Seitenumbruch\\
\end{tabbing}
\caption{Manueller Seitenumbruch}
\end{table}