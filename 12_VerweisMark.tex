\chapter{Verweise und Bemerkungen}
\section{Fußnoten}
\LaTeX{} stellt einen Befehl bereit, der automatisch Fußnoten erzeugt, nummeriert und den Text der Fußnote am Ende der Seite darstellt. Optional kann ein anderes Symbol zur Nummerierung vergeben werden. Mit \verb=\footnotemark= kann eine Fußnote ohne Text gesetzt werden (z.B. um auf eine bestehende Footmark per Label zu verweisen). Per \verb=\footnotetext= kann eine Fußnote gesetzt werden, welche nicht im Text referenziert wird.
\begin{lstlisting}[caption={Fußnoten (Befehle)}]
\footnote[nummer]{text}

\footnotemark[]

\footnotetext[]{}
\end{lstlisting}
\section{Randbemerkung}
\LaTeX{} unterstützt das setzten von Marginalien (Randbemerkungen). Hierbei wird der angegebene Text auf Höhe des Befehls an den Rand gesetzt.
\begin{lstlisting}[caption={Randbemerkung (Befehle)}]
Lorem ipsum dolor sit amet, consetetur sadipscing elitr, sed
diam nonumy eirmod\marginpar{Anmerkung} tempor invidunt ut
labore et dolore magna aliquyam erat, sed diam voluptua.
\end{lstlisting}
\section{Marken}
In \LaTeX{} kann mit Hilfe des Befehls \verb=\label{markenname}= eine Marke gesetzt werden. Zur Benennung sollten nur ASCII Zeichen verwendet werden, um Fehler beim kompilieren zu vermeiden. Markennamen müssen innerhalb eines Dokuments eindeutig sein.
\newpage
\section{Querverweise}
Sind Marken gesetzt worden können diese aufgerufen werden. Für die Bezugnahme stehen verschieden Befehle zur Verfügung. Querverweise werden in einer externen Datei gespeichert (\textit{.aux}). Mit Hilfe von Paketen kann weitere Funktionalität nachgeladen werden.
\begin{table}[H]
\begin{tabbing}
xxxxxxxxxxxxxxxxxxxxxxxxxxxxxx\=\kill
\verb=\ref{markenname}=				\>Kapitelnummer\\
\\
\verb=\pageref{markenname}=			\>Seitennummer\\
\\
\verb=\href{ziel}{text}=				\>Link ins Internet\\
\\
\verb=\hypertarget{name}{text}=			\>legt Ziel fest\\
\\
\verb=\hyperlink{ziel}{text}=				\>springt Ziel an\\
\end{tabbing}
\caption{Verweise (Befehle)}
\end{table}